\providecommand{\main}{..}
\documentclass[../main.tex]{subfiles}
\begin{document}
\chapter*{TODO: 大纲(草稿部分)}
\begin{multicols}{2}
\begin{enumerate}
    \item 引入部分
    \begin{enumerate}
        \item 线性方程组
        \item 尺规作图
        \item 五次方程
    \end{enumerate}
    \item 线性代数
    \begin{enumerate}
        \item 向量空间(还有例子们)
        \item 线性映射(相关命题和定理)
        \item 矩阵和行列式(相关命题和定理)
        \item 多重线性映射(对称、反对称、交替)
        \item 对偶空间
        \item 微小的应用:量子力学与量子计算
        \item 标准形和二次形
        \item 张量积(张量的坐标表示)
        \item 外代数
        \item Lie代数
        \item 含幺自同态的表示
    \end{enumerate}
    \item 群论
    \begin{enumerate}
        \item 群(和例子们)
        \item 子群和商群(和相关定理们)
        \item 自由群和自由阿贝尔群
        \item 群作用(Banach-Tarski定理?)
        \item Sylow定理
        \item Jordan-Holder定理
        \item Nielsen-Scheier定理(哈哈哈哈抄Rotman抄的真愉快)
        \item 有限群的表示
        \item 积和余积(范畴论初步)
        \item 同调群
        \item 导出函子($\mathbf{Ext}$ \& $\mathbf{Tor}$)
    \end{enumerate}
    \item Galois理论
    \begin{enumerate}
        \item 域论初步
        \item 例子:p-adic数域
    \end{enumerate}
    \item 环论和交换代数
    \begin{enumerate}
        \item 环
        \item 理想
        \item UFD、PID、欧几里得整环
        \item Noetherian环
        \item Artinian环
        \item Dedekind环
        \item 选择公理和Zorn引理
        \item 代数簇(越来越代数几何了起来)
    \end{enumerate}
    \item 范畴论和高阶范畴论
\end{enumerate}
\end{multicols}

\end{document}
