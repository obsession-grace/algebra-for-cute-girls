\documentclass[../main.tex]{subfiles}
\begin{document}
\chapter{群论}
\section{群}
\begin{exercise}\label{exe:order_and_cyclic_subgroup}
证明:若$g\in G$,则$g$的阶数是$|<g>|$。
\end{exercise}
\section{子群和商群}
\begin{exercise}\label{exe:coset_equirel_partition}
若$a\sim b$当且仅当存在某$x$使得$a,b\in xH$,证明$\sim$是一个等价关系。
\end{exercise}
\begin{proposition}{Lagrange定理}{lagrange}
若$H$是$G$的子群,且$G$是有限群,则$|H|$整除$G$。
\end{proposition}
\begin{exercise}
使用\ref{exe:coset_equirel_partition},证明Lagrange定理。提示:构造陪集间的双射。
\end{exercise}
\begin{proposition}{Cauchy定理}{cauchy}

\end{proposition}
\begin{theorem}{广义Lagrange定理}{generalized_lagrange}
若$H$是$G$的子群,存在双射$k:(G:H)\times H\bijection G$,使得$k([g],h)\subseteq[g]$
\end{theorem}
\section{交换群}
\section{自由群}
\section{群论主要结论}
\subsection{Sylow定理}
\begin{introduction}
\item 群的作用
\item $p$-子群
\item Sylow定理
\item Sylow定理的应用
\end{introduction}
\subsubsection{群的作用}
\begin{definition}{群的作用}{group_action}
若$G$是一群,$S$是一集合,称群同态$\pi:G\to\operatorname{Perm}(S)$为$G$在$S$上的一个\textbf{作用}。
\end{definition}
群的作用的另一种写法更加自然,我们写$g\cdot s=\pi(g)(s)$。
\begin{exercise}
请读者验证:
\begin{enumerate}
    \item $\forall x,y\in G,\,s\in S.\,(xy)\cdot s=x\cdot(y\cdot s)$;
    \item $\forall s\in S.\,1\cdot s=s$。
\end{enumerate}
反过来,若定义映射$\pi(g)(s)=g\cdot s$,则$\pi$是一个群作用。
\end{exercise}
因此,我们将交换使用这两种写法。另外若所讨论的作用根据上下文明确,我们将不会显式提及。
\begin{definition}{共轭}{group_conjugation}
对任意$x\in G$,定义$c_x:G\to G:y\mapsto xyx^{-1}$。请自行验证$c_{-}:G\to\automorphism(G)\subset\permutation(G)$为一同态。因此这样定义的$c$是一个$G$在$G$上的作用,被称为\textbf{共轭}。
\end{definition}
我们使用上标写法来书写共轭,即${}^xy=xyx^{-1}$和$y^x=x^{-1}yx$。$G$也在$G$的幂集上有作用:设$S=\mathcal{P}(G)$,若$A\in S$,则$xAx^{-1}\in S$。若存在$x\in G$使得$A,B\in S$满足$A=xBx^{-1}$,则称$A$和$B$共轭。
\begin{definition}{等方性群}{isotropy_group}
定义$\isotropy_G(s)=\{x\in G\,|\,x\cdot s=s\},$不难验证$\isotropy_G(s)\leq G$,被称为$s$在$G$中的\textbf{等方性群}。
\end{definition}
\begin{exercise}
证明:$G$作用于$S$,若$s,s'\in S$且存在$y\in G$使得$y\cdot s=s'$,那么$\isotropy_G(s)$与$\isotropy_{s'}(G)$共轭。
\end{exercise}
\begin{exercise}
证明:$\displaystyle K=\ker(\pi:G\to\operatorname{Perm}(S))=\bigcap_{s\in S}\isotropy_G(s)$
\end{exercise}
\begin{definition}{轨道}{action_orbit}
若$G$作用于$S$,定义$s$在$G$下的轨道为$\orbit_G(s)=\{xs\,|\,x\in G\}$。
\end{definition}
关于有限群$G$,我们有
\begin{proposition}{}{lang_5_1}
若$G$作用于$S$,$s\in S$,那么$|\orbit_G(s)|=(G:\isotropy_G(s))$。
\end{proposition}
\begin{proof}
定义$f(g)=g\cdot s$,则根据定义$\ker f=\isotropy_G(s)$。根据第一群同构定理(见下图),可得$G/\isotropy_G(s)\simeq\im f=\orbit_G(s)$,得证。
$$\begin{tikzcd}
G \arrow[rdd, "\varphi", hook] \arrow[rr, "f", two heads] & {} & \im f=\orbit_G(s) \\ {} & {} & {}\\ {} & G/\ker f=G/\isotropy_G(s) \arrow[ruu, "f_*", two heads, dotted, hook] & {}
\end{tikzcd}$$
\qed
\end{proof}
\begin{theorem}{轨道分解公式}{orbit_decomposition_formula}
若有限群$G$作用于$S$,$I$为一指标集,使得对于$i\neq j$有$\orbit_G(s_i)\cap\orbit_G(s_j)=\emptyset$,则有\textbf{轨道分解公式}$\displaystyle\operatorname{card}(S)=\sum_{i\in I}(G:\orbit_G(s_i))$。
\end{theorem}
\begin{proof}
首先证明有$\orbit_G(s_i)\cap\orbit_G(s_j)=\emptyset$或$\orbit_G(s_i)=\orbit_G(s_j)$成立。若$\orbit_G(s_i)\cap\orbit_G(s_j)=\{s,\dots\}$,则存在$x$使得$s=x\cdot s_i$,因此$\orbit_G(s)=\orbit_G(x\cdot s_i)=\orbit_G(s_i)$。类似地,我们有$\orbit_G(s)=\orbit_G(s_j)$,两轨道相等。因此,$S$是不同轨道的不交并,即$\displaystyle S=\coprod_{i\in I}\orbit_G(s_i)$。等式两边取基数,定理得证。\qed
\end{proof}
\begin{corollary}{共轭类公式}{class_formula}
在定理\ref{thm:orbit_decomposition_formula}中,取$S=G$,$G$以共轭作用于$G$本身,则轨道分解公式可被写成$\displaystyle|G|=\sum_{x\in C}(G:\orbit_G(x))$,其中$C$为不同共轭类的代表元。
\end{corollary}
\begin{proof}
若存在$g\in G$使得$a=gbg^{-1}$,则$(bgb^{-1}g^{-1})\cdot a=1\cdot b\in\orbit_G(a)\cap\orbit_G(b)$,根据定理\ref{thm:orbit_decomposition_formula}证明中论证,$\orbit_G(a)=\orbit_G(b)$。又由于共轭为一等价关系(请自行验证),对于任一轨道$\orbit_G(x)$,有$x\in[c]$,其中$c\in C$。设$I=C$,$s_i=i$,则$I$为满足定理\ref{thm:orbit_decomposition_formula}的指标集,推论得证。\qed
\end{proof}
\subsubsection{$p$-群}
\begin{definition}{$p$-群}{p_group}
若$p$为一个素数,且存在$n\geq0$使得$|G|=p^n$则称$G$为\textbf{$\mathbf{p}$-群};若$G$是某群的子群,则称其为\textbf{$\mathbf{p}$-子群}。
\end{definition}
\begin{proposition}{}{equidef_p_group}
$G$是一$p$-群当且仅当若对于每个$g\in G$,存在一个最小的$m\geq0$,使得$g^{(p^m)}=1$。
\end{proposition}
\begin{proof}
$(\implies)$对于任意的$g\in G$,其循环$\langle g\rangle$是$G$的一个子群。根据Lagrange定理(定理\ref{pro:lagrange}),$x=|\langle g\rangle|$整除$|G|$,由于$|G|=p^n$,$x=p^m$其中$0\leq m\leq n$。根据练习\ref{exe:order_and_cyclic_subgroup},$g$的阶数为$x$,即$p^m$,此方向得证。$(\impliedby)$用逆否命题,若$G$不为$p$-群,则$|G|$存在一个与$p$互素的不为$1$的因子$q$,根据命题\ref{pro:cauchy}(Cauchy定理),存在一个阶数为$q$的元素,此方向得证。\qed
\end{proof}
\begin{lemma}{$p$-子群引理}{lang_6_1}
\end{lemma}
\begin{lemma}{$p$-群不动点引理}{lang_6_3}
\end{lemma}
\subsubsection{Sylow定理}

\subsection{Schreier-Zassenhaus定理}
\subsection{Nielsen-Schreier定理}
\section{有限群的表示}
\section{范畴论初步}
\subsection{群的直积和直和}
\subsection{范畴和函子}
\subsection{积和余积}
\subsection{推出和拉回}
\section{同调论}
\subsection{同调群}
\subsection{欧拉示性数}
\subsection{分解}
\subsection{导出函子}
\subsection{谱序列}
\end{document}
