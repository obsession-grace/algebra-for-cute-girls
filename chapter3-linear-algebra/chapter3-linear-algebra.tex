\providecommand{\main}{..}
\documentclass[../main.tex]{subfiles}
\begin{document}
\setcounter{chapter}{2}
\chapter{线性代数}\label{cha:linear_algebra}
\section{线性代数与量子力学}
\subsection{*经典力学的回顾}
    量子力学的描写方法很多,与线性代数最为紧密的一种描写就是狄拉克符号了。为了更好地理解这种描写方式,我们首先来复习一下经典力学的部分和它相关的内容。
    考虑一个保守力场,我们知道:
    \begin{equation}
        \mathbf{F}= -\nabla V.
    \end{equation}
    通过牛顿第二定律定律可以得到运动方程:
    \begin{equation}
        m\frac{\text{d}\mathbf{v}}{\text{d}t} = -\frac{\partial  V}{\partial \mathbf{x}}.
    \end{equation}
    形式上我们也可以定义哈密顿量$H = \frac{\mathbf{p}^2}{2 m}+ V(\mathbf{x}) $ 来把运动定律写成哈密顿形式:

    \begin{eqnarray}
        \frac{\text{d}\mathbf{x}}{\text{d}t} &=&\frac{\partial  H}{\partial \mathbf{p}}, \\ \frac{\text{d}\mathbf{p}}{\text{d}t} &=& - \frac{\partial  H}{\partial \mathbf{x}}.  
    \end{eqnarray}
    一般地,我们可以用广义坐标$q_i$,广义动量$p_i$来代替公式里的$\mathbf{x}$和$\mathbf{p}$。到了这里,似乎还与我们的主题不太相关。接下来我们
\biblio
\end{document}
